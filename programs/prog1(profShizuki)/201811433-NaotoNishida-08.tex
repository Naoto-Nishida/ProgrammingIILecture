\documentclass[10pt,a4paper]{jsarticle}
\usepackage{listings}
\usepackage{fancyhdr}
\usepackage{lastpage}

\lhead{プログラミング実習1Bレポート(第8回)}
\rhead{学籍番号:201811433 氏名:西田 直人}
\cfoot{\thepage/\pageref{LastPage}}

\pagestyle{fancy}

\title{プログラミング1Bレポート課題第8回}
\author{西田直人}

\begin{document}
%\markright{プログラミング実習1Aレポート(第1回) 学籍番号:201811433 氏名:西田直人}
%\maketitle
\begin{center}
{\LARGE プログラミング1Bレポート課題第8回} \\
\large
西田直人 \\ 2018年10月23日
\end{center}
\normalsize
\section{課題8-1}
\noindent 1.クローズしないと書き出したつもりのデータもディスクに書き込まれる保証がない。\\
2.同時にオープンすることができるファイルの数には制限がある。
\section{課題8-2}
\subsection{a}
\subsubsection{sauce}
\lstinputlisting[basicstyle=\ttfamily\footnotesize,frame=single]{7-2a.c}

\subsubsection{result}
\begin{lstlisting}

\end{lstlisting}

\section{課題8-3}
\subsection{a}
\subsubsection{sauce}
\lstinputlisting[basicstyle=\ttfamily\footnotesize,frame=single]{report7-power.c}

\subsubsection{result}

\begin{lstlisting}

\end{lstlisting}

\subsection{b}
\subsubsection{sauce}
\lstinputlisting[basicstyle=\ttfamily\footnotesize,frame=single]{report7-length.c}

\subsubsection{result}
\begin{lstlisting}

\end{lstlisting}

\section{課題7-4}
\subsection{sauce}
\lstinputlisting[basicstyle=\ttfamily\footnotesize,frame=single]{report7-paren.c}

\subsection{result}

\begin{lstlisting}
  
\end{lstlisting}

\end{document}
